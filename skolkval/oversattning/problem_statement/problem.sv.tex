\problemname{Översättning}
\textit{Fikonspråket} är ett hemligt språk med gamla anor som till och med har givit upphov till vissa ord, 
till exempel \textit{fimp} (fikonspråkets \textit{fimpstukon} betyder \textit{stump}). 
\textit{Rövarspråket} användes av Kalle Blomkvist i Astrid Lindgrens böcker. 
Skriv ett program som översätter från rövarspråket till fikonspråket!

För båda språken gäller att man kan härleda varje ord från det motsvarande svenska ordet. 

\begin{itemize}
        \item För att översätta till fikonspråket delas ordet efter dess första \textit{vokal} i två delar. Dessa delar sätts sedan ihop i omvänd ordning och dessutom tillfogas \textbf{FI} i början och \textbf{KON} i slutet av ordet.
                \texttt{GET} blir alltså \texttt{FITGEKON}, 
                \texttt{LO} blir \texttt{FILOKON} och \texttt{ASTRONOM} blir \texttt{FISTRONOMAKON}.
        \item Översättning till rövarspråket sker bokstavsvis. 
                Vokaler ändras inte alls, medan varje konsonant skrivs två gånger med ett \textbf{O} emellan.
                \texttt{MAT} blir alltså \texttt{MOMATOT} och \texttt{ODLA} blir \texttt{ODODLOLA}.
\end{itemize}


Din uppgift är att skriva ett program som översätter ett ord skrivet på rövarspråk till fikonspråk.

\section*{Indata}
Den första och enda raden består av en sträng $S$, ett ord skrivet på rövarspråk. 
$S$ består endast av bokstäver \texttt{A}-\texttt{Z} och är max 30 tecken lång.
Det garanteras att det motsvarande svenska ordet till $S$ har minst en vokal.

\section*{Utdata}
Skriv ut en sträng, $S$ översatt till fikonspråk.

\section*{Poängsättning}
Din lösning kommer att testas på en mängd testfallsgrupper. För att få poäng för en grupp så måste du klara alla testfall i gruppen.


\noindent
\begin{tabular}{| l | l | p{12cm} |}
  \hline
  Grupp & Poängvärde & Gränser \\ \hline
  $1$   & $30$       & $S$ består bara av vokaler. \\ \hline
  $2$   & $70$       & Inga ytterligare begränsningar \\ \hline
\end{tabular}
