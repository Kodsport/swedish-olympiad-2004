\problemname{Översättning}
\textit{Fikon Språk} är ett hemligt språk med gamla anor som till och med har givit upphov till vissa ord, till exempel \textit{fimp} (fikonspråkets \textit{fimpstukon} betyder \textit{stump}). \textit{Rövarspråket} användes av Kalle Blomkvist i Astrid Lindgrens böcker. Skriv ett program som översätter från rövarspråket till fikonspråket!

För båda språken gäller att man kan härleda varje ord från det motsvarande ordet. 

\begin{itemize}
        \item För att översätta till fikonspråket delas ordet efter dess första \textit{vokal} i två delar. Dessa delar sätts sedan ihop i omvänd ordning och dessutom tillfogas \textbf{FI} i början och \textbf{KON} i slutet av ordet. \texttt{GET} blir alltså \texttt{FITGEKON}, \texttt{LO} blir \texttt{FILOKON} och \texttt{ASTRONOM} blir \texttt{FISTRONOMAKON}.
        \item Översättning till röverspråket sker bokstavsvis. Vokaler ändras inte alls, medan varje konsonant skrivs två gånger med ett \textbf{O} emellan. \texttt{MAT} blir alltså \texttt{MOMATOT} och \texttt{ODLA} blir \texttt{ODODLOLA}.
\end{itemize}

Programmet ska fråga efter ett ord ($\leq 30$ tecken), skrivet på rövarspråk. Endast versaler används vid inmatningen. Ordet innehåller bara bokstäver \textit{A}-\textit{Z} och motsvarande svenska ord har minst en vokal. Du kan förutsätta att ordet följer reglerna för rövarspråket. Sedan ska programmet skriva ut ordet översatt till fikonspråk.

\section*{Indata}
Den första linje består av ett heltal \textbf{N}, längden på ett sträng \textbf{S}, ett ord skrivet på rövarspråk.
Den andra linje består av strängen \textbf{S}. 

\section*{Utdata}
Skriv ut ett sträng, \textbf{S} översatt till fikonspråk.

\section*{Poängsättning}
Din lösning kommer att testas på en mängd testfallsgrupper. För att få poäng för en grupp så måste du klara alla testfall i gruppen.


\noindent
\begin{tabular}{| l | l | p{12cm} |}
  \hline
  Grupp & Poängvärde & Gränser \\ \hline
  $1$   & $30$       & S består bara av vokaler. \\ \hline
  $2$   & $70$       & Inga ytterligare begränsningar \\ \hline
\end{tabular}
