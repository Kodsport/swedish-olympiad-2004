\problemname{Pyramidtärningar}

I den här uppgiften använder vi tre tetraedrar, fyrsidiga kroppar, som tärningar. En kastad tetraeder-tärning läses av genom att lyfta upp den och ta reda på talet i botten. De tillsammans tolv sidorna har var och en ett unikt tal, $1...12$. Vi får en serie av 20 kast med de tre tärningarna och ska genom denna avgöra vika tal som finns på samma tärning. Tärningarna läses av i godtycklig ordning efter varje kast.

\section*{Indata}
Indatan består av 20 stycken tärningskast. Varje tärningskast står på en rad och består av tre tal mellan $1-12$. Indatan är utformad
sådan att det finns exakt en lösning.

\section*{Utdata}
Utdatan ska bestå av en rad för varje tärning. Raden ska innehålla fyra heltal - siffrorna på tärningen. Du måste skriva ut dess tal i sorterad ordning. Du ska först skriva ut den tärning som har siffran 1. Bland de två tärningar som är kvar ska du först skriva ut den med lägst minsta tal.

\section*{Poängsättning}
Din lösning kommer att testas på flera testfall. För att få 100 poäng så måste du klara alla testfall.
